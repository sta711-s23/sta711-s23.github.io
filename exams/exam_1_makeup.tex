\documentclass[11pt]{article}
\usepackage{url}
\usepackage{alltt}
\usepackage{bm}
\usepackage{bbm}
\linespread{1}
\textwidth 6.5in
\oddsidemargin 0.in
\addtolength{\topmargin}{-1in}
\addtolength{\textheight}{2in}

\usepackage{amsmath}
\usepackage{amssymb}

\begin{document}


\begin{center}
\Large
STA 711 Exam 1 Make-up\\
\normalsize
\vspace{5mm}
\end{center}

\noindent \textbf{Due:} Friday, March 3, 12:00pm (noon) on Canvas.\\ 

\noindent \textbf{Instructions:} Submit your work as a single PDF. For this assignment, you may include written work by scanning it and incorporating it into the PDF. Include all R code needed to reproduce your results in your submission.\\

\noindent \textbf{Rules:} This is a closed-book, closed-notes exam. You may:
\begin{itemize}
\item Email me, or come to office hours, with specific questions (I may be somewhat less helpful than for regular assignments)
\item Use the internet for general R help and debugging
\item Use R for question 7
\end{itemize}
You may \textit{not}:
\begin{itemize}
\item Use any resources from the course (the textbook, the course website, class notes, previous assignments, etc.)
\item Use the internet to look up any questions on the exam
\item Discuss the exam with anyone else
\end{itemize}

\newpage

\section*{Maximum likelihood and Fisher information}

\begin{enumerate}
\item Let $Y_1,...,Y_n$ be iid random variables with pdf
$$f(y|\mu, \sigma^2) = \frac{1}{y \sqrt{2 \pi \sigma^2}} \exp \left\lbrace -\frac{1}{2\sigma^2}(\log(y) - \mu)^2 \right\rbrace$$
where $\sigma^2 > 0$, $y > 0$, and $\mu \in \mathbb{R}$. Find the maximum likelihood estimators $\widehat{\mu}$ and $\widehat{\sigma}^2$.

\item Let $Y_1,...,Y_n \overset{iid}{\sim} Gamma(\alpha, \beta)$. Recall that the gamma distribution has pdf
$$f(y|\alpha, \beta) = \dfrac{\beta^\alpha}{\Gamma(\alpha)} y^{\alpha - 1} e^{-\beta y}$$
where $y > 0$, $\alpha > 0$, and $\beta > 0$. Find the Fisher information matrix $\mathcal{I}(\alpha, \beta)$. (You may assume that desired regularity conditions hold for the gamma distribution). Note: the derivative of the gamma function $\Gamma(x)$ has no nice closed form. You may use $\psi(x)$ to represent $\frac{d}{dx} \log \Gamma(x)$ (the digamma function), and $\psi_1(x)$ to represent $\frac{d}{dx} \psi(x)$ (the trigamma function).

\end{enumerate}

\newpage

\section*{Modeling an Exponential response}

In this section, you will model a response variable $Y_i \sim Exponential(\lambda_i)$.

\begin{enumerate}
\item[3.] Recall that if $Y \sim Exponential(\lambda)$, then the pdf of $Y$ is given by
$$f(Y;\lambda) = \frac{1}{\lambda} e^{-Y / \lambda}$$
Show that $\mathbb{E}[Y] = \lambda$ and $Var(Y) = \lambda^2$.

\item[4.] Suppose that $Y_i$ is a response variable of interest, and we use the following model for $Y_i$:
\begin{align*}
Y_i &\sim Exponential(\lambda_i) \\
\dfrac{1}{\lambda_i} &= \beta^T X_i,
\end{align*}
where $X_i = (1, X_{i,1},...,X_{i,k})^T \in \mathbb{R}^{k+1}$, and $\beta = (\beta_0, ..., \beta_k) \in \mathbb{R}^{k+1}$ is the vector of regression coefficients. We observe $n$ independent observations $(X_1, Y_1),...,(X_n, Y_n)$.

\begin{enumerate}
\item Derive a matrix form for the score $U(\beta)$.
\item Derive a matrix form for the Fisher information $\mathcal{I}(\beta)$. You may assume that desired regularity conditions hold.
\end{enumerate}

\item[5.] Now we apply the model from Question 4 to real data. A factory is interested in the relationship between the amount of stress applied to a piece of steel, and the time it takes until that steel breaks. We use the following model:
\begin{align*}
time_i &\sim Exponential(\lambda_i) \\
\dfrac{1}{\lambda_i} &= \beta_0 + \beta_1 stress_i
\end{align*}
The raw data contains $n = 40$ observations $(stress_1, \ time_1),...,(stress_{40}, \ time_{40})$.\\

\noindent You can load the data into R by
\begin{verbatim}
steel <- read.csv("https://sta711-s23.github.io/exams/steel.csv")
\end{verbatim}

In this question, we want to begin Fisher scoring to estimate $\beta_0$ and $\beta_1$ using the observed data.

\begin{enumerate}
\item In R, write functions \verb;U; and \verb;I; to calculate $U(\beta)$ and $\mathcal{I}(\beta)$ using the observed data.

\item Use your functions to estimate $\beta$ using Fisher scoring with the observed data. Iterate the Fisher scoring algorithm until convergence, beginning with $\beta^{(0)} = (1, 0)^T$. For the purpose of this question, stop when 
$$\max \{ |\beta_0^{(r+1)} - \beta_0^{(r)}|, \ |\beta_1^{(r+1)} - \beta_1^{(r)}| \} < 0.0001$$
\end{enumerate}

\end{enumerate}

\end{document}
