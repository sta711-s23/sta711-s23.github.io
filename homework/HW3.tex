\documentclass[11pt]{article}
\usepackage{url}
\usepackage{alltt}
\usepackage{bm}
\usepackage{bbm}
\linespread{1}
\textwidth 6.5in
\oddsidemargin 0.in
\addtolength{\topmargin}{-1in}
\addtolength{\textheight}{2in}

\usepackage{amsmath}
\usepackage{amssymb}

\begin{document}


\begin{center}
\Large
STA 711 Homework 3\\
\normalsize
\vspace{5mm}
\end{center}

\noindent \textbf{Due:} Friday, February 3, 12:00pm (noon) on Canvas.\\ 

\noindent \textbf{Instructions:} Submit your work as a single PDF. For this assignment, you may include written work by scanning it and incorporating it into the PDF. Include all R code needed to reproduce your results in your submission.

\section*{Maximum likelihood estimation}

\begin{enumerate}
\item Let $Y_1,...,Y_n$ be an iid sample from a distribution with pdf 
$$f(y | \lambda, \sigma) = \dfrac{\sigma^{1/\lambda}}{\lambda} \exp \left\lbrace -\left(1 + \frac{1}{\lambda}\right) \log(y) \right\rbrace \mathbbm{1}\{ y \geq \sigma \},$$
where $\lambda, \sigma > 0$. Find the maximum likelihood estimators of $\lambda$ and $\sigma$. \textit{(Hint: find $\widehat{\sigma}$ first)}
\end{enumerate}

\section*{Score and information}

\begin{enumerate}
\item[2.] Let $Y_1,...,Y_n \overset{iid}{\sim} Poisson(\lambda)$.
\begin{enumerate}
\item Find the score function $U(\lambda)$.
\item Calculate the Fisher information $\mathcal{I}(\lambda)$ using $Var(U(\lambda) | \lambda)$.
\item Calculate the Fisher information $\mathcal{I}(\lambda)$ using $-\mathbb{E}\left[ \dfrac{d^2}{d \lambda^2} \ell(\lambda | {\bf Y}) \ \biggr\lvert \ \lambda \right]$ (the required regularity conditions hold in this example).
\end{enumerate}

\item[3.] Consider a clinical trial to compare two treatments. $n_1$ subjects are given treatment 1, and $n_2$ subjects are given treatment 2. Let $Y_1$ be the number of people on treatment 1 who respond favorably, and $Y_2$ the number of people on treatment 2 who respond favorably. Assume that $Y_1 \sim Binomial(n_1, p_1)$ and $Y_2 \sim Binomial(n_2, p_2)$. The quantity of interest is the difference in the two treatments: $\psi = p_1 - p_2$.

\begin{enumerate}
\item Find the maximum likelihood estimate $\widehat{\psi}$ for $\psi$.

\item Since we have \textit{two} parameters, $p_1$ and $p_2$, Fisher information is no longer a scalar. Instead, $\mathcal{I}(p_1, p_2)$ is a $2 \times 2$ matrix. Under appropriate regularity conditions, it can be shown that the Fisher information matrix is given by
$$[\mathcal{I}(p_1, p_2)]_{ij} = - \mathbb{E}\left[ \frac{\partial^2}{\partial p_i \partial p_j} \ell(p_1, p_2 | {\bf Y}) \right].$$
Use this to calculate the Fisher information matrix $\mathcal{I}(p_1, p_2)$.

\item A sufficient condition for the formula in part (b) is given in Lemma 7.3.11 of Casella \& Berger, which essentially requires that we can differentiate under the integral sign. Read Section 2.4 of Casella \& Berger, on rules for differentiating under the integral sign. Then explain why the regularity conditions hold for this problem.
\end{enumerate}
\end{enumerate}

\section*{Fisher scoring problems}

In class, we learned how to use Fisher scoring to fit a logistic regression model. Recall that the Fisher scoring algorithm estimates the parameters $\beta$ of a model as follows:

\begin{itemize}
\item Start with an initial guess $\beta^{(0)}$
\item Update the estimate: $\beta^{(r+1)} = \beta^{(r)} + \mathcal{I}^{-1}(\beta^{(r)}) U(\beta^{(r)})$
\item Stop when $\beta^{(r+1)} \approx \beta^{(r)}$
\end{itemize}

\noindent The purpose of these questions is to practice with Fisher scoring.\\

\begin{enumerate}

\item[4.] In this problem, we will work with the dengue data we discussed in class. A CSV containing the data can be downloaded in \texttt{R} by running
\begin{center}
\texttt{dengue <- read.csv("https://sta711-s23.github.io/homework/dengue.csv")}
\end{center}

For this problem, we are interested in modeling the relationship between platelet count and dengue fever. Let $PLT_i$ denote the platelet count of patient $i$, and $Y_i$ denote their dengue status (0 = negative, 1 = positive). Our logistic regression model is

\begin{align*}
Y_i &\sim Bernoulli(p_i)\\
\log \left( \dfrac{p_i}{1 - p_i} \right) &= \beta_0 + \beta_1 PLT_i
\end{align*}

\begin{enumerate}
\item Fit this logistic regression model in R, and report the estimated coefficients $\widehat{\beta}_0$ and $\widehat{\beta}_1$.
\item In R, write a function \texttt{U} which calculates $U(\beta)$ using the \texttt{dengue} data. For example, if $\beta = (1.8, 0)^T$ then your function should produce the following:

\begin{verbatim}
U(c(1.8, 0))
[1]   -3211.612 -820195.802
\end{verbatim}

\item In R, write a function \texttt{I} which calculates $\mathcal{I}(\beta)$ using the \texttt{dengue} data. For example, if $\beta = (1.8, 0)^T$ then your function should produce the following:

\begin{verbatim}
> I(c(1.8, 0))
            [,1]       [,2]
[1,]    696.2918   161214.3
[2,] 161214.2603 41783775.1
\end{verbatim}

\item Suppose that we use Fisher scoring to estimate $\beta$, and our current estimate is $\beta^{(r)} = (1.8, 0)^T$. Calculate the updated estimate $\beta^{(r+1)}$.

\item Use your code from (b) and (c) to write code which implements Fisher scoring until convergence, beginning with $\beta^{(0)} = (1.8, 0)^T$. For the purpose of this question, stop when 
$$\max \{ |\beta_0^{(r+1)} - \beta_0^{(r)}|, \ |\beta_1^{(r+1)} - \beta_1^{(r)}| \} < 0.0001$$
Does your final estimate match the estimated coefficients in (a)? How many scoring iterations did it take to converge?

\end{enumerate}

\item[5.] One alternative to Fisher scoring is \textit{gradient ascent}, variations of which are often used to fit complicated machine learning models for which it is challenging to calculate the Hessian / Fisher information. Rather than the Fisher information, gradient ascent uses a \textit{learning rate} (or \textit{step size}) $\gamma > 0$ to update coefficient estimates. 

\begin{itemize}
\item Start with an initial guess $\beta^{(0)}$
\item Update the estimate: $\beta^{(r+1)} = \beta^{(r)} + \gamma U(\beta^{(r)})$
\item Stop when $\beta^{(r+1)} \approx \beta^{(r)}$
\end{itemize}

\begin{enumerate}
\item Modify your code from 4(e) to implement gradient ascent instead of Fisher scoring. Use a learning rate (step size) of $\gamma = 0.0000001$, begin with $\beta^{(0)} = (1.8, 0)^T$, and run for 5000 iterations (do not run until convergence!). Report the estimated coefficients after 5000 steps. Why do you think Fisher scoring performs better here than gradient ascent?
\end{enumerate}

\item[6.] So far, we have applied Fisher scoring to estimate parameters in logistic regression models. How does this relate to estimation for \textit{linear} regression models?\\

Consider the model
\begin{align*}
Y_i &\sim N(\mu_i, \sigma^2) \\
\mu_i &= \beta^T X_i
\end{align*}
where $\beta = (\beta_0, \beta_1, ..., \beta_k)^T$ and $X_i = (1, X_{i,1},...,X_{i,k})^T$. Suppose we observe data $(X_1,Y_1),...,(X_n,Y_n)$, and we want to estimate $\beta$.

\begin{enumerate}
\item Write down the log likelihood function $\ell(\beta | {\bf X}, {\bf Y})$.

\item Show that the score function, in matrix form, is given by
$$U(\beta) = \frac{1}{\sigma^2} {\bf X}^T ({\bf Y} - \bm{\mu}),$$
where $\bm{\mu} = {\bf X} \beta$.

\item Set the score equal to 0 and solve for $\beta$ to get
$$\widehat{\beta} = ({\bf X}^T {\bf X})^{-1} {\bf X}^T {\bf Y}$$

\item Show that the Hessian of the log likelihood, in matrix form, is given by
$${\bf H}(\beta) = -\frac{1}{\sigma^2} {\bf X}^T {\bf X}$$

\item As we can see from (c), for \textit{linear} regression we can get a closed form for $\widehat{\beta}$. But for the sake of comparison with logistic regression, let's suppose instead that we use Fisher scoring. Let $\beta^{(0)}$ be \textit{any} initial estimate of $\beta$. Show that the result from a single iteration of Fisher scoring is
$$\beta^{(1)} = \widehat{\beta} = ({\bf X}^T {\bf X})^{-1} {\bf X}^T {\bf Y}$$
(in other words, Fisher scoring converges in a single step).
\end{enumerate}

\end{enumerate}

\section*{Sneak peek: Poisson regression}

\begin{enumerate}
\item[7.] So far, we have worked with logistic regression models for a binary response. In STA 712, we will work with other types of response variables, like a Poisson response. This question will give you a preview of Poisson regression, while giving you practice with Fisher scoring. \\

Consider the Poisson regression model
\begin{align*}
Y_i &\sim Poisson(\lambda_i) \\
\log(\lambda_i) &= \beta^T X_i,
\end{align*}
where $\beta = (\beta_0, \beta_1, ..., \beta_k)^T$ and $X_i = (1, X_{i,1},...,X_{i,k})^T$. Suppose we observe data $(X_1, Y_1),...,(X_n, Y_n)$, where $X_i = (1, X_{i,1},...,X_{i,k})^T \in \mathbb{R}^{k+1}$. (Since $\lambda > 0$ for a Poisson variable, $\log(\lambda) \in (-\infty, \infty)$, which makes it reasonable for $\log(\lambda_i)$ to be a linear function of the $X$s).

\begin{enumerate}
\item Show that the score function is
\begin{align*}
U(\beta) = \mathbf{X}^T({\bf Y} - \bm{\lambda}),
\end{align*}
where $\bm{\lambda} = (\lambda_1,...,\lambda_n)^T$.

\item Let ${\bf W} = \text{diag}(\lambda_1,...,\lambda_n)$, where $\lambda_i = \exp\{ \bm{\beta}^T X_i \}$. By calculating the variance of $U(\beta)$, show that the Fisher information is
\begin{align*}
\mathcal{I}(\beta) = {\bf X}^T {\bf W} {\bf X}.
\end{align*}

\item In R, simulate $n = 500$ observations $(X_1, Y_1),...,(X_n, Y_n)$. Draw $X_{i,1} \overset{iid}{\sim} N(0, 1)$, and $Y_i \sim Poisson(\lambda_i)$, where $\log(\lambda_i) = -2 + 2 X_{i,1}$. 

\item Using your simulated data from part (c), fit a Poisson regression model of $Y$ on $X$, and report the fitted model coefficients. To fit a Poisson regression model in R: 

\begin{verbatim}
glm(y ~ x, family = poisson)
\end{verbatim}

\item Modify your code from question 4 to implement Fisher scoring for Poisson regression with the simulated data. Begin with $\beta^{(0)} = (0, 0)^T$, and stop when
$$\max \{ |\beta_0^{(r+1)} - \beta_0^{(r)}|, \ |\beta_1^{(r+1)} - \beta_1^{(r)}| \} < 0.0001$$
Does your final estimate match the estimated coefficients in (d)? How many scoring iterations did it take to converge?
\end{enumerate}

\end{enumerate}

\end{document}
