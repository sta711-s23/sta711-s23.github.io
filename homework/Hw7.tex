\documentclass[11pt]{article}
\usepackage{url}
\usepackage{alltt}
\usepackage{bm}
\usepackage{bbm}
\linespread{1}
\textwidth 6.5in
\oddsidemargin 0.in
\addtolength{\topmargin}{-1in}
\addtolength{\textheight}{2in}

\usepackage{amsmath}
\usepackage{amssymb}

\begin{document}


\begin{center}
\Large
STA 711 Homework 7\\
\normalsize
\vspace{5mm}
\end{center}

\noindent \textbf{Due:} Monday, March 27, 12:00pm (noon) on Canvas.\\ 

\noindent \textbf{Instructions:} Submit your work as a single PDF. For this assignment, you may include written work by scanning it and incorporating it into the PDF. Include all R code needed to reproduce your results in your submission.


\begin{enumerate}
\item Let $X_1,...,X_n$ be an iid sample from a population with mean $\mu$ and variance $\sigma^2$, and suppose that $\sigma^2$ is known. We wish to test the hypotheses $H_0: \mu = \mu_0$ vs. $H_A: \mu \neq \mu_0$.

\begin{enumerate}
\item Write an expression for the (approximate) power function for the Wald test of these hypotheses.
\item Plot power as a function of $\mu$, using $\alpha = 0.05$, $\mu_0 = 0$, $n = 100$, and $\sigma^2 = 1$.
\item Let $\beta(\mu)$ be the (approximate) power function for the Wald test. Show mathematically that for each $\mu \neq \mu_0$, $\beta(\mu) \to 1$ as $n \to \infty$.
\item Suppose that $\alpha = 0.05$, $\mu_0 = 0$ and $\sigma^2 = 1$. What is the minimum sample size $n$ needed such that $\beta(0.5) > 0.7$?
\end{enumerate}

\item Suppose that $X_1,...,X_n \overset{iid}{\sim} N(0, \sigma^2)$. We wish to test the hypotheses $H_0: \sigma^2 = \sigma_0^2$ vs. $H_A: \sigma^2 = \sigma_1^2$, were $\sigma_0^2 < \sigma_1^2$.

\begin{enumerate}
\item Show that the most powerful test of these hypotheses rejects when $\sum \limits_{i=1}^n X_i^2 > c$, for some value $c$.
\item Find $c$ such that the test in part (a) has size $\alpha$.
\end{enumerate}

\item Suppose that $X_1,...,X_n \overset{iid}{\sim} N(\mu, \sigma^2)$, with both $\mu$ and $\sigma^2$ unknown. Our hypotheses are $H_0: \sigma^2 = \sigma_0^2$ vs. $H_A: \sigma^2 \neq \sigma_0^2$. Propose a test statistic and rejection region for testing these hypotheses, such that the resulting test is size $\alpha$.

\item Suppose that $X_1,...,X_n \overset{iid}{\sim} Pareto(\theta, \nu)$, with pdf
$$f(x |\theta, \nu) = \dfrac{\theta \nu^{\theta}}{x^{\theta + 1}} \mathbbm{1}\{x \geq \nu\},$$
where $\theta, \nu > 0$. 

\begin{enumerate}
\item Find the maximum likelihood estimators of $\theta$ and $\nu$.

\item We wish to test $H_0: \theta = 1$ vs. $H_A: \theta \neq 1$, and $\nu$ is unknown. The likelihood ratio test rejects when
$$\dfrac{\sup \limits_{\theta > 0, \nu > 0} L(\nu, \theta | {\bf X})}{\sup \limits_{\theta = 1, \nu > 0} L(\nu, \theta | {\bf X})} > k.$$
Show that the likelihood ratio test is equivalent to rejecting when $T \leq c_1$ or $T \geq c_2$, where $0 < c_1 < c_2$ and
$$T = \log \left( \frac{\prod \limits_{i=1}^n X_i}{X_{(1)}^n} \right).$$
\end{enumerate}

\item Suppose we have two independent samples $X_1,...,X_n \overset{iid}{\sim} Exponential(\theta)$ and $Y_1,...,Y_m \overset{iid}{\sim} Exponential(\mu)$. The likelihood ratio test rejects when 
$$\frac{\sup \limits_{\theta > 0, \mu > 0} L(\theta, \mu | {\bf X})}{\sup \limits_{\theta = \mu} L(\theta, \mu | {\bf X})} > k.$$
Show that the LRT can be based on the statistic
$$T = \frac{\sum \limits_{i=1}^n X_i}{\sum \limits_{i=1}^n X_i + \sum \limits_{j=1}^m Y_j}.$$

\item (Global $F$-test for linear regression) Suppose that $V_1 \sim \chi^2_{d_1}$ and $V_2 \sim \chi^2_{d_2}$ are independent $\chi^2$ random variables. Then $F = \dfrac{V_1/d_1}{V_2/d_2} \sim F_{d_1, d_2}$, where $F_{d_1, d_2}$ denotes the $F$-distribution with numerator degrees of freedom $d_1$ and denominator degrees of freedom $d_2$.\\

The $F$-distribution is important for hypothesis testing in linear regression models. Suppose we observe independent data $(X_1, Y_1),...,(X_n, Y_n)$, where $Y_i = \beta^T X_i + \varepsilon_i$, with $\beta = (\beta_0, ..., \beta_k)^T$ and $\varepsilon_i \overset{iid}{\sim} N(0, \sigma^2)$. We wish to test the hypotheses
$$H_0: \beta_1 = \cdots = \beta_k = 0 \hspace{1cm} H_A: \text{at least one of } \beta_1,...,\beta_k \neq 0.$$
The $F$-test for these hypotheses is based on the $F$-statistic
$$F = \dfrac{(SSTO - SSE)/k}{SSE/(n - k - 1)},$$
where $F \sim F_{k, n-k-1}$ under $H_0$, and
\begin{align*}
SSTO = \sum \limits_{i=1}^n (Y_i - \overline{Y})^2 \hspace{1cm} SSE = \sum \limits_{i=1}^n (Y_i - \widehat{\beta}^T X_i)^2
\end{align*}

The goal of this problem is to demonstrate that, indeed, $F \sim F_{k, n-k-1}$ under $H_0$.

\begin{enumerate}
\item Show that under $H_0$, $\frac{1}{\sigma^2}\sum \limits_{i=1}^n (Y_i - \beta_0)^2 \sim \chi^2_n$.

\item Find symmetric matrices $A_1, A_2, A_3$ such that under $H_0$, 
$$\frac{1}{\sigma^2}\sum \limits_{i=1}^n (Y_i - \beta_0)^2 = Z^T A_1 Z + Z^T A_2 Z + Z^T A_3 Z$$
where $Z \sim N(0, I)$, $\frac{1}{\sigma^2} SSE = Z^T A_1 Z$, and $\frac{1}{\sigma^2} (SSTO - SSE) = Z^T A_2 Z$.

\item Using the matrices $A_1, A_2, A_3$ from part (b), show that $rank(A_1) = n-k-1$, $rank(A_2) = k$, and $rank(A_3) = 1$.

\item By applying Cochran's theorem, show that $F = \dfrac{(SSTO - SSE)/k}{SSE/(n - k - 1)} \sim F_{k, n-k-1}$ under $H_0$.
\end{enumerate}
\end{enumerate}


\end{document}
